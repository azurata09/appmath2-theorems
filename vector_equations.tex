\section{ベクトル方程式}

\begin{theorem}
  ベクトル$\vec a, \vec b$が垂直のとき,$\vec a \cdot \vec b = 0$である.
\end{theorem}

\begin{proof}
  \[
    \vec a \cdot \vec b = |\vec a| \, |\vec b| \cos \frac{\pi}{2} = 0 (\because \cos \frac{\pi}{2} = 0)
  \]
\end{proof}

\begin{theorem}
  $a, b, c$を定数とし,$xy$平面上において$ax + by = c$と表わされる直線$l$の単位法線ベクトルは,$\displaystyle (\pm \frac{a}{\sqrt{a^2 + b^2}}, \pm \frac{b}{\sqrt{a^2 + b^2}})$であり,原点からの距離は$\displaystyle \frac{|c|}{\sqrt{a^2 + b^2}}$である.
\end{theorem}

\begin{proof}
  直線$l$の単位法線ベクトルを$\vec n = (n_x, n_y)$,原点からの距離を$\rho$,$l$上の動点$P$の位置ベクトルを$\vec r = (x, y)$とおくと,$\vec n \cdot \vec r = \rho$となる.

  これを成分表示すると$n_x x + n_y y = \rho$であり,これはヘッセの標準形である.

  ヘッセの標準形であるならば$||\vec n|| = 1$,つまり${n_x}^2 + {n_y}^2 = 1$である.これを$ax + by = c$と比較すると単位法線ベクトルは次の通りである.

\[
  (\pm \frac{a}{\sqrt{a^2 + b^2}}, \pm \frac{b}{\sqrt{a^2 + b^2}})
\]

  また,原点からの距離は次の通りである.

\[
  \frac{|c|}{\sqrt{a^2 + b^2}}
\]

\end{proof}

ただし,これではあまりにもわかりづらいので,別の証明も記す.

\begin{proof}
  $ax + by = c$をヘッセの標準形に変換する.

  例えば,$Ax + By = C$がヘッセの標準形であるとき,$A^2 + B^2 = 1$である\footnote{ヘッセの標準形は$x \cos\theta + y \sin\theta = 1$であるが,$\cos^2 \theta + \sin^2 \theta = 1$である.}.$a^2 + b^2 = M$とするとき,$M$を$1$にするには,両辺を$M$で割ればよい.つまり,次のようになる.

  \[
    a^2 + b^2 = M
  \]
  \[
    \frac{a^2}{M} + \frac{b^2}{M} = 1
  \]
  \[
    \frac{a^2}{a^2 + b^2} + \frac{b^2}{a^2 + b^2} = 1 (\because M = a^2 + b^2)
  \]

  すなわち,$ax + by = c$をヘッセの標準形に直すと次の通りである.

  \[
    \frac{a}{\sqrt{a^2 + b^2}} x + \frac{b}{\sqrt{a^2 + b^2}} y = \frac{c}{\sqrt{a^2 + b^2}}
  \]

  ヘッセの標準形$Ax + By = C$において,$A, B$はそれぞれ単位法線ベクトルの$x$成分$y$成分であるから,単位法線ベクトルは次の通りである.

  \[
    (\pm \frac{a}{\sqrt{a^2 + b^2}}, \pm \frac{b}{\sqrt{a^2 + b^2}})
  \]

  また,ヘッセの標準形$Ax + By = C$において,$C$は原点からの距離にあたるが,距離が$0$未満になることはありえないため,絶対値をつけて次の通りとなる.

  \[
    \frac{|c|}{\sqrt{a^2 + b^2}}
  \]
\end{proof}

\begin{theorem}
  $y = \alpha x$が$x$軸となす角は$\arctan \alpha$である.
\end{theorem}

\begin{proof}
  $y=\tan\theta x$を下に図示する.

  \begin{center}
    \begin{tikzpicture}
      % x軸
      \draw[->,>=stealth,semithick](-2,0)--(2,0)node[above]{$x$};

      % y軸
      \draw[->,>=stealth,semithick](0,-1)--(0,3)node[right]{$y$};%y軸

      % 原点
      \draw(0,0)node[above left]{O};%原点


      % 点(1, \tan\theta)
      \draw(1,{sqrt(3)})node[right]{$(1, tan\theta)$};
      \fill(1,{sqrt(3)})circle(2pt);%原点

      \draw[dashed](1,0)--(1,{sqrt(3)});
      \draw(1,{sqrt(3)/2})node[right]{$\tan\theta$};
      \draw(1/2,0)node[below]{$1$};

      % \draw plot(\x,{sqrt(3)\x});
    \end{tikzpicture}
  \end{center}

  $\tan\theta = \alpha$とすると,$y = \alpha x$が$x$軸となす角は$\arctan \alpha$である.
\end{proof}
